\section{Equations}                                                      
In this section, we present the equation that models the transport of
electrons. We will also show why it is so important to use Galerkin
quadratures. First, we present the 
Boltzmann-Fokker-Planck equation (BFP). The idea is to decompose the highly forward
peaked scattering cross section into a sum of a forward-peaked cross section
and a smoothly varying cross section. The BFP equation is given by (the variables are
omitted for brevity) \cite{morel_96} :
\begin{equation}
\begin{split}
\bo \cdot \bn \Psi + \Sigma_{t} \Psi &= \int_{4\pi}\int_0^{\infty} \Sigma_{s}\(\bo
\cdot \bo',E'\rightarrow E\) \Psi(\bo',E')dE'd\bo'\\
& + \frac{\alpha}{2}
\(\frac{\partial}{\partial \mu} \(1-\mu^2\) \frac{\partial \Psi}{\partial \mu}
+ \frac{1}{1-\mu^2} \frac{\partial^2 \Psi}{\partial \phi^2}\) +
\frac{\partial S\Psi}{\partial E}  + Q
\end{split}
\label{bfp}
\end{equation}
where : 
\begin{itemize}
\item $\Psi$ is the angular flux 
\item $\Sigma_t$ is the smooth-component of the total macroscopic cross section
\item $\Sigma_s$ is the smooth-component of the macroscopic differential scattering cross section
\item Q is a volumetric source
\item $\mu$ is the cosine of the directional polar angle
\item $\phi$ is the directional azimuthal angle
\end{itemize} 
The second and the third term of the 
right-hand-side change the direction of the particles without changing their energy 
and the fourth term changes the energy of the particles without changing their 
direction (continuous slowing-down term) \cite{morel_81}. The continuous slowing 
down is used for the ``soft" interactions that result in small-energy losses. The 
catastrophic interactions that result in large energy losses are represented with 
the standard Boltzmann operator. $\alpha$ is the restricted momentum transfer :
\begin{equation}
\alpha(E) = 2 \pi \int_0^E \int_{-1}^1 \Sigma_{ss}(E\rightarrow E',\mu_0) (1-\mu_0) 
d\mu_0 dE
\end{equation}
with $\mu_0 = \mu'\mu+\sqrt{\(1-\mu'^2\)\(1-\mu^2\)} cos\(\phi'-\phi\)$ and
$\Sigma_{ss}(E\rightarrow E', \mu_0)$ denotes the forward-peaked scattering
cross section.\\
$S$ is the restricted stopping power :
\begin{equation}
S(E) = 2\pi \int_0^{E} \int_{-1}^{1} \Sigma_{ss}(E\rightarrow E',\mu_0) \(E-E'\)d\mu_0 dE'
\end{equation}
The restricted stopping power is defined as the portion of the total stopping
power that is not due to catastrophic collisions.\\
All possible standard boundary conditions can be applied to (\ref{bfp}), the
most likely being the vacuum boundary conditions :
\begin{equation}
\Psi(\br,\bo,E) = 0 \ \ \ \textrm{ for } \bo \cdot \bs{n} < 0 \textrm{ and } \br
\in \partial \mathcal{D}_v
\end{equation}
and the incoming flux boundary conditions :
\begin{equation}
\Psi(\br,\bo,E) = g(\br,\bo,E)  \ \ \ \textrm{ for } \bo \cdot \bs{n} < 0 \textrm{ and } \br
\in \partial \mathcal{D}_i
\end{equation}
Where $\partial \mathcal{D}_v$ is the boundary of the domain where vacuum
conditions are applied and $\partial \mathcal{D}_i$ is the boundary of the
domain where incoming flux conditions are applied.\\
In this work, we will use another approach and
retain only the asymptotic limit on the energy dependence but not the angular
dependence. We will keep the continuous slowing-down term and approximate the
forward peaked angular dependence by a Dirac distribution. The equation can be written as : 
\begin{equation}
\bo \cdot \bn \Psi + \Sigma_{t} \Psi = \int_{4\pi}\int_0^{\infty}
\Sigma_{s}\(\bo
\cdot \bo',E'\rightarrow E\) \Psi(\bo',E')dE'd\bo' + \int_{-1}^1 c
\delta(\mu-1) \Psi d\mu + \frac{\partial S\Psi}{\partial E} + Q
\label{solved}
\end{equation}
The continuous slowing-down term in the 
equation (\ref{solved}) can be either treated explicitly or it can be treated in 
the cross sections. The latter was done in the CEPXS
code which produces cross sections for photons and electrons transport \cite{cepxs}. 
In this work, we use cross sections produced by CEPXS to compute the coupled 
photon-electron transport. Unlike \cite{acuros}, we do not assume that the electrons 
do not produce photons. The system is fully coupled, meaning that photons produce 
electrons and electrons produce photons. Thus, there are some upscattering 
terms present in the scattering matrix even if there is not upscattering
physically. The upscattering is between particle types; a given particle can
only loses energy and creates others particle types which are represented by
the upscattering terms.\\
Next, we focus on the scattering term, assuming that this quantity has been
integrated over the energy \hbox{range :}
\begin{equation}
R = \int_{4\pi} \Sigma_{s} \(\bo \cdot \bo'\) \Psi\(\bo'\) d\bo' + \int_{-1}^1
c\delta (\mu-1) \Psi d\mu
\label{scattering}
\end{equation}
In a $S_n$ code, it is usual to write the discretized scattering source 
(\ref{scattering}) as a product of 3 \hbox{matrices \cite{graal} :}
\begin{equation}
\bs{R} = M \Sigma D \bs{\Psi}
\end{equation}
where $\bs{R}$ is the vector containing the scattering source, $\bs{\Psi}$ is
the vector containing all the flux moments, $D$ is the discrete-to-moment matrix 
which maps a vector of discrete angular flux values to a corresponding vector of flux
moments, $\Sigma$ is the scattering matrix which contains the moments of the 
scattering cross sections on its diagonal, and $M$ is the moment-to-discrete matrix 
which maps a vector of flux moments to a corresponding vector of discrete angular 
flux values. The Galerkin quadratures require that :
\begin{equation}
D = M^{-1}
\end{equation}
and, therefore, $M$ and $D$ have to be square matrices. This implies that the number of
moments is equal to the number of directions.\\
Galerkin quadratures integrate exactly a delta
function scattering and that is why we need to use them. To see this, assume that :
\begin{equation}
\Sigma(\mu) = \delta(\mu-1)
\end{equation}
It is obvious that :
\begin{equation}
\begin{split}
R &= \int_{-1}^1 \delta(\mu-1) \Psi(\mu) d\mu\\
&= \Psi
\end{split}
\end{equation}
Because $P_l(1)=1$ for all $l$, all of the Legendre expansion coefficients, $\Sigma_l$, for the 
delta function equal to unity and the cross-section matrix is the identity
matrix. We get :
\begin{equation}
\bs{R} = MD\bs{\Psi}= \bs{\Psi}
\end{equation}

