\section{Equations}
The transport equation for neutral particle is given by :
\begin{equation}
\bo \cdot \bn \Psi(\br,\bo,E) + \Sigma_t(\br,E) \Psi(\br,\bo,E) =
\int_{4\pi}\int_0^{\infty} \Sigma_s\(\br,\bo \cdot \bo',E'\rightarrow E\)
\Psi(\br,\bo',E')dE'd\bo' + Q(\br,\bo,E)
\end{equation}
Like we said before, we will replace this equation by the
Boltzmann-Fokker-Planck equation (BFP). The idea is to decompose the highly forward
peaked scattering cross section into a sum of a forward-peaked cross section
and a smooth cross section. The BFP equation is given by (the variables are
omitted for brevity) \cite{morel_96} :
\begin{equation}
\bo \cdot \bn \Psi + \Sigma_t \Psi = \int_{4\pi}\int_0^{\infty} \Sigma_s\(\bo
\cdot \bo',E'\rightarrow E\) \Psi(\bo',E')dE'd\bo' + \frac{\alpha}{2}
\(\frac{\partial}{\partial \mu} \(1-\mu^2\) \frac{\partial \Psi}{\partial \mu}
+ \frac{1}{1-\mu^2} \frac{\partial^2 \Psi}{\partial \phi}\) +
\frac{\partial S\Psi}{\partial E}  + Q
\end{equation}
where the second and the third term of the right-hand-side changes the direction of 
the particles without changing the energy and the fourth term changes the energy of 
the particles without changing the direction (continuous slowing-down term).
The continuous slowing down is used for the "soft" interactions that result in
small-energy losses. The catastrophic interactions that result in large energy
losses are represented with the standard Boltzmann scattering. $\alpha$ is the 
restricted momentum transfer :
\begin{equation}
\alpha(E) = 2 \pi \int_0^E \int_{-1}^1 \Sigma_s(E\rightarrow E',\mu_0) (1-\mu_0) 
d\mu_0 dE
\end{equation}
and $S$ is the restricted stopping power :
\begin{equation}
S(E) = 2\pi \int_0^{E} \int_{-1}^{1} \Sigma_s(E\rightarrow E') \(E-E'\)d\mu_0 dE'
\end{equation}
The restricted stopping power is defined as the portion of the total stopping
power which is not due to catastrophic collisions.\\
It is worth to notice that the term :
$\frac{\alpha}{2}\(\frac{\partial}{\partial \mu}\(1-\mu^2\)\frac{\partial
\Psi}{\partial \mu}+\frac{1}{1-\mu^2}\frac{\partial^2\Psi}{\partial \phi}\)$ is very 
close to 0 when $\mu \neq 1$. Thus, this term can be replace by
$\int_{-1}^1 c\delta(\mu-1) \Psi d\mu$. So the equation can be rewritten as : 
\begin{equation}
\bo \cdot \bn \Psi + \Sigma_t \Psi = \int_{4\pi}\int_0^{\infty} \Sigma_s\(\bo
\cdot \bo',E'\rightarrow E\) \Psi(\bo',E')dE'd\bo' + \int_{-1}^1 c
\delta(\mu-1) \Psi d\mu + \frac{\partial}{\partial E}\beta \Psi + Q
\label{solved}
\end{equation}
The 2 extra-terms, in the previous equation, can be either treated explicitly
or they can be treated in the cross sections. That was done in the CEPXS code
\cite{cepxs}. In this work, we will use the cross section produced by CEPXS to
compute the coupled photon-electron transport. Unlike \cite{acuros} we do not 
assume that the electrons do not produce photons. The system is fully coupled 
(photons produce electrons and electrons produce photons) and thus, there is some 
upscattering term in the scattering matrix even if there is not upscattering 
physically.\\
Now we will focus on the scattering term assuming that this quantity has been
integrated over the \hbox{energy :}
\begin{equation}
S = \int_{4\pi} \Sigma_s \(\bo \cdot \bo'\) \Psi\(\bo'\) d\bo' + \int_{-1}^1
c\delta (\mu-1) \Psi d\mu
\label{scattering}
\end{equation}
In a $S_n$ code, it is usual to write the discretized scattering source 
(\ref{scattering}) a product of 3 matrices \cite{graal} :
\begin{equation}
\bs{S} = M \Sigma D \bs{\Psi}
\end{equation}
where $\bs{\Psi}$ is a vector containing all the flux moments, $D$ is the
discrete-to-moment matrix which maps a vector of discrete angular flux values
to a corresponding vector of flux moments, $\Sigma$ is the scattering matrix
which contains the moments of the scattering cross sections on its diagonal and 
$M$ is the moment-to-discrete matrix which maps a vector of flux moments to a 
corresponding vector of discrete angular flux values. The Galerkin quadrature 
requires that :
\begin{equation}
M = D^{-1}
\end{equation}
and thus, $M$ and $D$ has to be square, which implies that the number of
moments is equal to number of direction.\\
We want to use the Galerkin quadrature because it integrates exactly a delta
function scattering. Let's assume that :
\begin{equation}
\Sigma(\mu) = \delta(\mu-1)
\end{equation}
It is obvious that :
\begin{equation}
\begin{split}
S &= \int_{-1}^1 \delta(\mu-1) \Psi(\mu) d\mu\\
&= \Psi
\end{split}
\end{equation}
Because $P_l(1)=1$ for all $l$, all expansion coefficients fort the delta
function are equal to unity, and the cross-section matrix is the identity
matrix. Thus, we have :
\begin{equation}
\begin{split}
\bs{S} &= MD\bs{\Psi}\\
&= \bs{\Psi}\\
\end{split}
\end{equation}

