\begin{abstract}
In this work, we present two methods to decrease memory consumption while solving 
the photon-electron transport equation. The coupled transport of electrons and
photons is very important in radiotherapy because it models the interactions of 
the X-rays. One of the issues of discretized electron transport is that the
scattering of electrons is very forward peaked. A common approximation is to
represent that peak in the scattering cross section by a Dirac distribution.
This is convenient, but the integration of this distribution requires the
use of the Galerkin quadrature. By construction this quadrature imposes that
the number of moments of the flux equals the number of directions 
(number of angular fluxes), which is very demanding in term of memory. In this study, 
we show that even if the number of moments is not as large as the number 
of directions, we can obtain an accurate solution. Another method to decrease the 
memory consumption involves choosing an appropriate order for the energy
groups that we use. 
We show in this paper that an appropriate alternation of photons/electrons groups 
allows us to rewrite one transport problem of $n$ groups as $gcd$ successive 
transport problems of $\frac{n}{gcd}$ groups. Where $gcd$ is the greatest common 
divisor between the number of groups of photons and the number of groups of
electrons.
\keywords{Galerkin quadrature, photon-electron transport, radiotherapy}
\end{abstract}
