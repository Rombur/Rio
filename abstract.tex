\begin{abstract}
In this work, we present two methods to decrease memory consumption while solving 
the Boltzmann-Fokker-Planck equation. The Boltzmann-Fokker-Planck equation is
the relation used to described the transport of electron. The transport of
electron, coupled with the transport of photon, is very important in
radiotherapy because it models the interactions between the X-rays and the
human body. One of the problems of electron transport is that the
scattering of electrons is very forward. A very common approximation is to
represent that peak in the scattering cross section by a Dirac distribution.
This is convenient but the integration of this distribution requires the
use of the Galerkin quadrature. This quadrature imposes that the number of moments of
the flux to be equal to the number of directions (equal to the number of angular 
fluxes), which is very demanding in term of memory. In this study, 
we show that even if we do not keep the number of moments as high as the number 
of directions, we can keep an accurate solution. Another method to decrease the 
memory consumption consists in choosing an appropriate order for the energy groups. 
We show in this paper that an appropriate alternation of photons/electrons groups 
allows us to rewrite one transport problem of $n$ groups as $gcd$ successive 
transport problems of $\frac{n}{gcd}$ groups. Where $gcd$ is the greatest common 
divisor between the number of groups of photon and the number of groups of electron.
\keywords{Galerkin quadrature, electron, radiotherapy}
\end{abstract}
