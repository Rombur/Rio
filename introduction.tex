\section{Introduction}
The transport of photons and electrons has many applications in medical
physics and particularly in radiotherapy. Radiotherapy uses photons or charged
particles to damage the DNA of the cancerous cells. While using photons, free 
electrons are created which ionize the environment and create free radicals. 
These free radicals damage the DNA of the cells. One quantity used to gauge whether 
a cell will die due to radiation is the absorbed dose. The absorbed dose is 
defined as the energy deposition by unit of mass, measured in unit of Gray 
$\(Gy=\frac{J}{kg}\)$. This is the quantity that will be determined by the physician 
in order to treat the tumor. To compute this dose, several methods can be applied : 
semi-analytic methods, deterministic methods and Monte-Carlo methods.
Monte-Carlo methods are very interesting to have very accurate 
results, however they are slow to converge and they remain too slow for 
effective clinical use \cite{acuros}. Other methods like pencil-beam convolution and
convolution superposition employ pre-calculated Monte-Carlo dose
kernels, which are then locally scaled to approximate photon and electron
transport in the presence of heterogeneities. These methods have some problems in 
the presence of large density gradients, such as those at interfaces between 
different materials: air, bone, lung and soft tissue \cite{acuros} \cite{seco}
\cite{krieger}. Therefore, the deterministic methods appear to be more and more 
interesting for radiotherapy. It has been shown that there is agreement between 
deterministic methods and Monte-Carlo methods \cite{acuros}.\\
In this work, we will present a $S_n$ method for the photon-electron
transport. The difficulty of this calculation comes from the transport of the
electrons. Because the electrons are charged particles, they have very
anisotropic scattering due to their interactions with other particles through 
Coulomb interaction. This anisotropy causes some
complications since the standard Legendre expansion representing the cross-sections 
would require hundreds of terms. A common approximation is to use a Dirac
distribution to model the forward-peaked scattering of the electrons 
and a continuous slowing down term for energy loss due to Coulomb
interactions. This allows the Legendre expansion of the cross-section to 
keep a low order. However, an exact integration of the Dirac distribution requires 
to use the Galerkin quadrature \cite{graal}, which demands a lot of memory. This 
quadrature needs the number of flux moments and the number of angular fluxes to be 
equal and this number rises very quickly with the order of the $S_n$ method. 
For example, when using the $S_n$ discretization for $n=16$, we have, in 3D, 
288 directions and thus, 288 angular fluxes or flux moments to store. In this 
work, we will see the effects of truncation on the Galerkin quadrature. We will 
change the order of the scattering cross sections and observed the effect on the 
absorbed dose. The goal is to keep as few flux moments as possible while 
maintaining an accurate solution. We will also show a method of ordering
the energy groups which allows to decompose one big transport problem in
several smaller transport problems. This interesting possibility is due to
the fact that there is no thermalization of the particles. Therefore, every
particle undergoes only slowing down and the scattering matrix can be written
as a lower block triangular matrix. The interactions between photons and
electrons forbids a lower triangular matrix by adding some
upscattering terms in the scattering matrix.
