\section{Introduction}
The transport of photons and electrons has a lot of applications in medical
physics and particularly in radiotherapy. The radiotherapy consists of sending
X-rays on a cancerous tumor in order to destroy it. The photons interact in the
body producing electrons by photo-electric effect. These electrons destroy the
cells and produce secondary photons which will also produce electrons. One
quantity used to know if a cell will die or due to radiation is the 
absorbed dose. The absorbed dose is defined as the energy deposition by unit of mass. 
The unit of the absorbed dose is the Gray $\(Gy=\frac{J}{kg}\)$. This is the
quantity that will be chosen by the physician in order to kill the tumor. The dose 
has to be large enough in the tumor to kill the cancerous cells but as small as 
possible in the healthy cells. It is very important to spare the healthy cells
because the radiation can cause cancer several years later. To compute this
dose, several methods can be applied :\\
\begin{description}
\item[Semi-analytic methods;] very fast but inaccurate.
\item[Deterministic methods;] fast and accurate but require a lot of memory.
\item[Monte-Carlo methods;] very accurate but slow to converge.
\end{description}
Monte-Carlo methods are, of course, very interesting to have very accurate 
results, however they are slow to converge and they remain too slow for 
effective clinical use \cite{acuros}. Other methods like pencil-beam convolution and
convolution superposition employ the use of pre-calculated Monte-Carlo dose
kernels, which are then locally scaled to approximate photon and electron
transport in the presence of heterogeneities. These methods have some trouble in 
the presence of large density gradients, such as those at interfaces between 
different materials: air, bone, lung and soft tissue \cite{acuros}. Therefore, 
the deterministic methods appear to be more and more interesting for radiotherapy. 
It has been shown that a good agreement can be obtained between 
deterministic methods and Monte-Carlo methods \cite{acuros}.\\
In this work, we will present a $S_n$ method for the photon-electron
transport. The difficulty of this calculation comes from the transport of the
electrons. The electrons are charged particles and thus, they have very
anisotropic scattering because they can interact with other particles through 
Coulomb interaction. The electrons have a lot of interactions where the energy 
and the direction almost do not change. This anisotropy creates some problems 
since the standard Legendre expansion representing the cross-sections would 
require hundreds of terms. A common approximation is to replace the Boltzmann 
equation by the Boltzmann-Fokker-Planck equation. These equation consists of 
adding two terms: one causes to particles to redistribute in direction without 
energy change and the other to redistribute in energy without directional 
change \cite{morel_81}. This allows the Legendre expansion of the cross-section to 
keep a low order. It is worth to note that the Henyey-Greenstein kernel widely used 
to describe the scattering of light in biological tissues cannot be expressed using 
the BFP equation \cite{larsen}. However, this model, even if it is useful, because 
of its simplicity, has no theoretical root. The reason why the Henyey-Greenstein 
cannot be expressed with the BFP equation is that the BFP equation assumes that the 
scattering cross section is highly peaked. The Henyey-Greenstein cross section is 
not peaked enough to be represented by the BFP equation.\\ 
The last approximation that we will make is to replace the term modifying the
direction without changing the energy by a Dirac distribution. However, an exact 
integration of the Dirac distribution requires to use the Galerkin quadrature 
\cite{graal}, which requires a lot of memory. This quadrature needs the number of 
flux moments and the number of angular fluxes to be equal. This number rises very 
quickly with the order the $S_n$ method. For example, when using the $S_n$ 
discretization for $n=16$, we have, in 3D, 288 directions and thus, 288 angular 
fluxes or flux moments to store. In this work, we will see the effects of truncation 
on the Galerkin quadrature. The goal is to keep as few flux moments as possible while 
still keeping an accurate solution. We will also show a method of ordering
the energy groups which allows to decompose one big transport problem in
several smaller transport problems.
