\section{Introduction}
The transport of photons and electrons has many applications in medical
physics and particularly in radiotherapy. Radiotherapy uses photons and charged
particles to damage the DNA of cancerous cells. While using photons, free 
electrons are created; this ionizes the environment and creates free radicals
that damage the cells DNA. One quantity used to gauge whether 
a cell will die due to radiation is the absorbed dose, 
defined as the energy deposited per unit of mass and measured in Gray 
$\(Gy=\frac{J}{kg}\)$. Several methods can be applied to solve the
photon-electron distribution in the body : 
semi-analytic methods, deterministic methods and Monte-Carlo methods.
Monte-Carlo methods yield very accurate 
results, however they are slow to converge and thus, remain too slow for 
effective clinical use \cite{acuros}. Other methods, such as pencil-beam convolution and
convolution-superposition, employ pre-calculated Monte-Carlo dose
kernels, which are then locally scaled to approximate photon and electron
transport in the presence of heterogeneities. These methods present some issues in 
the presence of large density gradients, such as those found at interfaces between 
different materials: air, bone, lung and soft tissue \cite{acuros,seco,krieger}. 
Therefore, deterministic methods appear to be more and more 
interesting for radiotherapy. It has been shown that there is agreement between 
deterministic methods and Monte-Carlo methods \cite{acuros}.\\
In this work, we present a $S_n$ method for photon-electron
transport. The difficulty of this calculation comes from the transport of the
electrons. Because the electrons are charged particles, they have very
anisotropic scattering due to their interactions with other particles through 
Coulomb interaction. This anisotropy causes some
complications since the standard Legendre expansion representing the cross sections 
would require hundreds of terms. A common approximation is to use a Dirac
distribution to model the forward-peaked scattering of the electrons 
and a continuous slowing down term for energy loss due to Coulomb
interactions. This allows the Legendre expansion of the cross section to 
be kept to a low order. However, an exact integration of the Dirac distribution requires 
to use Galerkin quadratures \cite{graal}, which demand a lot of memory. These
quadratures need the number of flux moments and the number of angular fluxes to be 
equal and this number varies as $\frac{n(n+2)}{8}$ per octant with the order of the $S_n$ method.  
For example, when using the $S_n$ discretization for $n=16$, we have, in 3D, 
288 directions and thus, 288 angular fluxes or flux moments to store. In this 
work, we analyze the effects of truncation on the Galerkin quadrature. We
alter the order of the scattering cross sections and observed the effect on the 
absorbed dose. The goal is to keep as few flux moments as possible while 
maintaining an accurate solution. We also show a method of ordering
the energy groups which allows to decompose one large transport problem in
several smaller transport problems. This interesting possibility is due to
the fact that there is no thermalization of the particles. Therefore, every
particle undergoes only slowing down and the scattering matrix can be written
as a lower block triangular matrix. The interactions between photons and
electrons forbids a lower triangular matrix by adding some
upscattering terms in the scattering matrix. The code generating the cross
sections, CEPXS \cite{cepxs}, generates first the cross sections for one
particle type and then, the cross sections for the other one. We show in this
paper how to improve the order of the groups.
