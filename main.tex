\documentclass{article}
\usepackage{amsmath}
\usepackage{array}
\usepackage{color}
\usepackage{graphicx}
\usepackage{float} %utiliser H pour forcer � mettre l'image o� on veut
\usepackage{lscape} %utilisation du mode paysage
\usepackage{mathbbol} % permet d'avoir le vrai symbol pour les reels grace a mathbb
\usepackage{enumerate}
\usepackage{marvosym}	
\usepackage{moreverb} % permet d'utiliser verbatimtab : conservation la tabulation 


\setlength {\textwidth}{16cm}
\setlength {\textheight}{21cm} 
\setlength {\oddsidemargin}{0cm}
\setlength{\headsep}{5pt} 

\newcommand\bn{\boldsymbol{\nabla}}
\newcommand\bo{\boldsymbol{\Omega}}
\newcommand\br{\mathbf{r}}
\newcommand\la{\left\langle}
\newcommand\ra{\right\rangle}
\newcommand\bs{\boldsymbol}
\newcommand\red{\textcolor{red}}

\renewcommand{\(}{\left(}
\renewcommand{\)}{\right)}
\renewcommand{\[}{\left[}
\renewcommand{\]}{\right]}

\newtheorem{theorem}{Theorem}[section]

\begin{document}
\title{Coupled photon-electron transport}
\author{Bruno Turcksin \and Jean Ragusa \and Jim Morel \and Damien
Lebrun-Grandie} 
\date{}
\maketitle

% Introduction
\section{Introduction}
The transport of photon and electron has a lot of applications in medical
physics and particularly in radiotherapy. The radiotherapy consists in sending
X-rays on a cancerous tumor in order to destroy it. The photons interact in the
body producing electrons by photo-electric effect. These electrons destroy the
cells but produce secondary photons which will also produce electron. One
quantity used to know if a cell will die or not is the absorbed dose. The
absorbed dose is defined as the energy deposition by unit of mass. The unit of the 
absorbed dose is the Gray ($GY=\frac{J}{kg}$). It is important to know accurately what
is the dose sent. The dose has to be large enough in the tumor to kill the
cancerous but as small as possible in the healthy cells. To compute this dose,
several methods can be applied :
\begin{description}
\item[Semi-analytic methods,] very fast but inaccurate.
\item[Deterministic method,] fast and accurate but needs a lot of memory.
\item[Monte-Carlo,] very accurate but slow to converge.
\end{description}
The goal of the radiotherapy is to sent a dose as high as possible while
sparing the healthy cells. It is very important to spare the healthy cells
because the radiation can cause cancer several years later. Thus, it is more
and more important to know the dose accurately. Monte-Carlo method are of
course a very interesting method to have very accurate results however they
are slow to converge. Nowadays, they remain too slow for effective clinical
use \cite{acuros}. Others methods are pencil-beam convolution and
convolution superposition employ the use of pre-calculated Monte-Carlo dose
kernels, which are then locally scaled to approximate photon and electron
transport in the presence of heterogeneities. These methods, however, have
some troubles in the presence of large density gradients, such as those at
interfaces between different materials : air, bone, lung and soft tissue 
\cite{acuros}. That is because of all the previous reasons that the
deterministic methods appear to be more and more interesting for radiotherapy. 
It has been shown in the past  that a very good agreement can be obtained
between deterministic method and Monte-Carlo method \cite{acuros}.\\
In this work, we will present a $S_n$ method for the photon-electron
transport. The difficulty of this calculation comes from the transport of the
electrons. The electrons are charged particles and thus, they have very
anisotropic scattering. This is due to the fact that can interact with others
particles through Coulomb interaction. Thus, the electrons have a lot of
interactions where the energy and the direction almost do not change. This 
anisotropy creates some problems since the standard Legendre expansion to represent 
the cross-sections would require hundreds of terms. A common approximation is to 
replace the Boltzmann equation by the Boltzmann-Fokker-Planck equation. These 
equation consists in adding in 2 terms: one causes to particles to redistribute in 
direction without energy change and the other to redistribute in energy without 
directional change\cite{morel_81}. This allows to keep the Legendre expansion of the 
cross-section to a low order. It is worth to note that the Henyey-Greenstein kernel
used to describe the scattering of light in biological tissues cannot be
expressed using the BFP equation \cite{larsen}. However, this model, even if
it is useful, because of its simplicity, has no theoretical root. The reason
is that the BFP equation assume that the scattering cross section is very
peaked. The Henyey-Greenstein is not peaked enough to be represented by the
BFP equation. We will see that it is very convenient to rewrite the term changing 
the direction as a Dirac distribution. However, an exact integration requires then to use the Galerkin quadrature \cite{graal}, which requires a lot of memory. 
Effectively, this quadrature needs that the number of flux moments and the
number of angular fluxes are equal. This number rises very quickly with the
order the $S_n$ method. For example, when using the $S_n$ discretization for $n=16$, 
we have, in 3D, 288 directions and thus, 288 angular fluxes or flux moments to
store. In this work, we will see the effects of truncation on the Galerkin
quadrature. The goal is to keep as few flux moments as possible while still
keeping an accurate solution.


% Equations
\section{Equations}                                                      
In this section, we present the equation that models the transport of
electrons. We will also show why it is so important to use Galerkin
quadratures. First, we present the 
Boltzmann-Fokker-Planck equation (BFP). The idea is to decompose the highly forward
peaked scattering cross section into a sum of a forward-peaked cross section
and a smoothly varying cross section. The BFP equation is given by (the variables are
omitted for brevity) \cite{morel_96} :
\begin{equation}
\begin{split}
\bo \cdot \bn \Psi + \Sigma_{t} \Psi &= \int_{4\pi}\int_0^{\infty} \Sigma_{s}\(\bo
\cdot \bo',E'\rightarrow E\) \Psi(\bo',E')dE'd\bo'\\
& + \frac{\alpha}{2}
\(\frac{\partial}{\partial \mu} \(1-\mu^2\) \frac{\partial \Psi}{\partial \mu}
+ \frac{1}{1-\mu^2} \frac{\partial^2 \Psi}{\partial \phi^2}\) +
\frac{\partial S\Psi}{\partial E}  + Q
\end{split}
\label{bfp}
\end{equation}
where : 
\begin{itemize}
\item $\Psi$ is the angular flux 
\item $\Sigma_t$ is the smooth-component of the total macroscopic cross section
\item $\Sigma_s$ is the smooth-component of the macroscopic differential scattering cross section
\item Q is a volumetric source
\item $\mu$ is the cosine of the directional polar angle
\item $\phi$ is the directional azimuthal angle
\end{itemize} 
The second and the third term of the 
right-hand-side change the direction of the particles without changing their energy 
and the fourth term changes the energy of the particles without changing their 
direction (continuous slowing-down term) \cite{morel_81}. The continuous slowing 
down is used for the ``soft" interactions that result in small-energy losses. The 
catastrophic interactions that result in large energy losses are represented with 
the standard Boltzmann operator. $\alpha$ is the restricted momentum transfer :
\begin{equation}
\alpha(E) = 2 \pi \int_0^E \int_{-1}^1 \Sigma_{ss}(E\rightarrow E',\mu_0) (1-\mu_0) 
d\mu_0 dE
\end{equation}
with $\mu_0 = \mu'\mu+\sqrt{\(1-\mu'^2\)\(1-\mu^2\)} cos\(\phi'-\phi\)$ and
$\Sigma_{ss}(E\rightarrow E', \mu_0)$ denotes the forward-peaked scattering
cross section.\\
$S$ is the restricted stopping power :
\begin{equation}
S(E) = 2\pi \int_0^{E} \int_{-1}^{1} \Sigma_{ss}(E\rightarrow E',\mu_0) \(E-E'\)d\mu_0 dE'
\end{equation}
The restricted stopping power is defined as the portion of the total stopping
power that is not due to catastrophic collisions.\\
Standard boundary conditions can be applied to (\ref{bfp}), the
most likely being the vacuum boundary conditions :
\begin{equation}
\Psi(\br,\bo,E) = 0 \ \ \ \textrm{ for } \bo \cdot \bs{n} < 0 \textrm{ and } \br
\in \partial \mathcal{D}_v
\end{equation}
and the incoming flux boundary conditions :
\begin{equation}
\Psi(\br,\bo,E) = g(\br,\bo,E)  \ \ \ \textrm{ for } \bo \cdot \bs{n} < 0 \textrm{ and } \br
\in \partial \mathcal{D}_i
\end{equation}
where $\partial \mathcal{D}_v$ is the boundary of the domain where vacuum
conditions are applied and $\partial \mathcal{D}_i$ is the boundary of the
domain where incoming flux conditions are applied.\\
In this work, we will use another approach and
retain only the asymptotic limit on the energy dependence but not the angular
dependence. We will keep the continuous slowing-down term and approximate the
forward peaked angular dependence by a Dirac distribution. The equation can be written as : 
\begin{equation}
\bo \cdot \bn \Psi + \Sigma_{t} \Psi = \int_{4\pi}\int_0^{\infty}
\Sigma_{s}\(\bo
\cdot \bo',E'\rightarrow E\) \Psi(\bo',E')dE'd\bo' + \int_{-1}^1 c
\delta(\mu-1) \Psi d\mu + \frac{\partial S\Psi}{\partial E} + Q
\label{solved}
\end{equation}
The continuous slowing-down term in the 
equation (\ref{solved}) can be either treated explicitly or it can be treated in 
the cross sections. The latter was done in the CEPXS
code which produces cross sections for photons and electrons transport \cite{cepxs}. 
In this work, we use cross sections produced by CEPXS to compute the coupled 
photon-electron transport. Unlike \cite{acuros}, we do not assume that the electrons 
do not produce photons. The system is fully coupled, meaning that photons produce 
electrons and electrons produce photons. Thus, there are some upscattering 
terms present in the scattering matrix even if there is not upscattering
physically. The upscattering is between particle types; a given particle can
only loses energy and creates others particle types which are represented by
the upscattering terms.\\
Next, we focus on the scattering term, assuming that this quantity has been
integrated over the energy \hbox{range :}
\begin{equation}
R = \int_{4\pi} \Sigma_{s} \(\bo \cdot \bo'\) \Psi\(\bo'\) d\bo' + \int_{-1}^1
c\delta (\mu-1) \Psi d\mu
\label{scattering}
\end{equation}
In a $S_n$ code, it is usual to write the discretized scattering source 
(\ref{scattering}) as a product of 3 \hbox{matrices \cite{graal} :}
\begin{equation}
\bs{R} = M \Sigma D \bs{\Psi}
\end{equation}
where $\bs{R}$ is the vector containing the scattering source, $\bs{\Psi}$ is
the vector containing all the flux moments, $D$ is the discrete-to-moment matrix 
which maps a vector of discrete angular flux values to a corresponding vector of flux
moments, $\Sigma$ is the scattering matrix which contains the moments of the 
scattering cross sections on its diagonal, and $M$ is the moment-to-discrete matrix 
which maps a vector of flux moments to a corresponding vector of discrete angular 
flux values. The Galerkin quadratures require that :
\begin{equation}
D = M^{-1}
\end{equation}
and, therefore, $M$ and $D$ have to be square matrices. This implies that the number of
moments is equal to the number of directions.\\
Galerkin quadratures integrate exactly a delta
function scattering and that is why we need to use them. To see this, assume that :
\begin{equation}
\Sigma(\mu) = \delta(\mu-1)
\end{equation}
It is obvious that :
\begin{equation}
\begin{split}
R &= \int_{-1}^1 \delta(\mu-1) \Psi(\mu) d\mu\\
&= \Psi
\end{split}
\end{equation}
Because $P_l(1)=1$ for all $l$, all of the Legendre expansion coefficients, $\Sigma_l$, for the 
delta function equal to unity and the cross-section matrix is the identity
matrix. We get :
\begin{equation}
\bs{R} = MD\bs{\Psi}= \bs{\Psi}
\end{equation}



\bibliographystyle{unsrt}
\bibliography{biblio}

\end{document}
