\documentclass[11pt,letter,twoside]{mc2011}
\usepackage{amsmath}
\usepackage{array}
%\usepackage{color}
\usepackage{graphicx}
\usepackage{float} %utiliser H pour forcer � mettre l'image o� on veut
\usepackage{lscape} %utilisation du mode paysage
\usepackage{mathbbol} % permet d'avoir le vrai symbol pour les reels grace a mathbb
\usepackage{enumerate}
\usepackage{marvosym}	
\usepackage{moreverb} % permet d'utiliser verbatimtab : conservation la tabulation 
\usepackage{lineno}


\setlength {\textwidth}{16cm}
\setlength {\textheight}{21cm} 
\setlength {\oddsidemargin}{0cm}
\setlength{\headsep}{5pt} 

\newcommand\bn{\boldsymbol{\nabla}}
\newcommand\bo{\boldsymbol{\Omega}}
\newcommand\br{\mathbf{r}}
\newcommand\la{\left\langle}
\newcommand\ra{\right\rangle}
\newcommand\bs{\boldsymbol}
\newcommand\red{\textcolor{red}}

\renewcommand{\(}{\left(}
\renewcommand{\)}{\right)}
\renewcommand{\[}{\left[}
\renewcommand{\]}{\right]}

\newtheorem{theorem}{Theorem}[section]

\usepackage[numbers,sort&compress]{natbib}

\usepackage{fancyhdr,lastpage}

%\usepackage[dvips]{graphicx,color}

\pagestyle{fancy}

\globalmc2011

\begin{document}
%\linenumbers
\title{To reduce memory needs for coupled photon-electron transport.}
\author{
    \textbf{Bruno Turcksin, Jean Ragusa, and Jim Morel}\\
    Texas A\&M University, Department of Nuclear Engineering\\
    College Sation, Texas 77843-3133\\
    turcksin@neo.tamu.edu; ragusa@ne.tamu.edu; morel@ne.tamu.edu
}
\date{}
\maketitle

\thispagestyle{empty}

% Abstract
\begin{abstract}
In this work, we present two methods to decrease memory consumption while solving 
the Boltzmann-Fokker-Planck equation. The Boltzmann-Fokker-Planck equation is
the relation used to described the transport of electron. The transport of
electron, coupled with the transport of photon, is very important in
radiotherapy because it models the interactions between the X-rays and the
human body. One of the problems of electron transport is that the
scattering of electrons is very forward. A very common approximation is to
represent that peak in the scattering cross section by a Dirac distribution.
This is convenient but the integration of this distribution requires the
use of the Galerkin quadrature. This quadrature imposes that the number of moments of
the flux to be equal to the number of directions (equal to the number of angular 
fluxes), which is very demanding in term of memory. In this study, 
we show that even if we do not keep the number of moments as high as the number 
of directions, we can keep an accurate solution. Another method to decrease the 
memory consumption consists in choosing an appropriate order for the energy groups. 
We show in this paper that an appropriate alternation of photons/electrons groups 
allows us to rewrite one transport problem of $n$ groups as $gcd$ successive 
transport problems of $\frac{n}{gcd}$ groups. Where $gcd$ is the greatest common 
divisor between the number of groups of photon and the number of groups of electron.
\keywords{Galerkin quadrature, electron, radiotherapy}
\end{abstract}


\newcommand\authorname{Bruno Turcksin, Jean Ragusa, and Jim Morel}
\newcommand\shorttitlename{Methods to reduce memory needs for photon-electron transport.}

\fancymc2011

% Introduction
\section{Introduction}
The transport of photons and electrons has a lot of applications in medical
physics and particularly in radiotherapy. The radiotherapy consists of sending
X-rays on a cancerous tumor in order to destroy it. The photons interact in the
body producing electrons by photo-electric effect. These electrons destroy the
cells and produce secondary photons which will also produce electrons. One
quantity used to know if a cell will die or due to radiation is the 
absorbed dose. The absorbed dose is defined as the energy deposition by unit of mass. 
The unit of the absorbed dose is the Gray $\(Gy=\frac{J}{kg}\)$. This is the
quantity that will be chosen by the physician in order to kill the tumor. The dose 
has to be large enough in the tumor to kill the cancerous cells but as small as 
possible in the healthy cells. It is very important to spare the healthy cells
because the radiation can cause cancer several years later. To compute this
dose, several methods can be applied :\\
\begin{description}
\item[Semi-analytic methods;] very fast but inaccurate.
\item[Deterministic methods;] fast and accurate but require a lot of memory.
\item[Monte-Carlo methods;] very accurate but slow to converge.
\end{description}
Monte-Carlo methods are, of course, very interesting to have very accurate 
results, however they are slow to converge and they remain too slow for 
effective clinical use \cite{acuros}. Other methods like pencil-beam convolution and
convolution superposition employ the use of pre-calculated Monte-Carlo dose
kernels, which are then locally scaled to approximate photon and electron
transport in the presence of heterogeneities. These methods have some trouble in 
the presence of large density gradients, such as those at interfaces between 
different materials: air, bone, lung and soft tissue \cite{acuros}. Therefore, 
the deterministic methods appear to be more and more interesting for radiotherapy. 
It has been shown that a good agreement can be obtained between 
deterministic methods and Monte-Carlo methods \cite{acuros}.\\
In this work, we will present a $S_n$ method for the photon-electron
transport. The difficulty of this calculation comes from the transport of the
electrons. The electrons are charged particles and thus, they have very
anisotropic scattering because they can interact with other particles through 
Coulomb interaction. The electrons have a lot of interactions where the energy 
and the direction almost do not change. This anisotropy creates some problems 
since the standard Legendre expansion representing the cross-sections would 
require hundreds of terms. A common approximation is to replace the Boltzmann 
equation by the Boltzmann-Fokker-Planck equation. These equation consists of 
adding two terms: one causes to particles to redistribute in direction without 
energy change and the other to redistribute in energy without directional 
change \cite{morel_81}. This allows the Legendre expansion of the cross-section to 
keep a low order. It is worth to note that the Henyey-Greenstein kernel widely used 
to describe the scattering of light in biological tissues cannot be expressed using 
the BFP equation \cite{larsen}. However, this model, even if it is useful, because 
of its simplicity, has no theoretical root. The reason why the Henyey-Greenstein 
cannot be expressed with the BFP equation is that the BFP equation assumes that the 
scattering cross section is highly peaked. The Henyey-Greenstein cross section is 
not peaked enough to be represented by the BFP equation.\\ 
The last approximation that we will make is to replace the term modifying the
direction without changing the energy by a Dirac distribution. However, an exact 
integration of the Dirac distribution requires to use the Galerkin quadrature 
\cite{graal}, which requires a lot of memory. This quadrature needs the number of 
flux moments and the number of angular fluxes to be equal. This number rises very 
quickly with the order the $S_n$ method. For example, when using the $S_n$ 
discretization for $n=16$, we have, in 3D, 288 directions and thus, 288 angular 
fluxes or flux moments to store. In this work, we will see the effects of truncation 
on the Galerkin quadrature. The goal is to keep as few flux moments as possible while 
still keeping an accurate solution. We will also show a method of ordering
the energy groups which allows to decompose one big transport problem in
several smaller transport problems.


% Equations
\section{Equations}
The transport equation for neutral particle is given by :
\begin{equation}
\bo \cdot \bn \Psi(\br,\bo,E) + \Sigma_t(\br,E) \Psi(\br,\bo,E) =
\int_{4\pi}\int_0^{\infty} \Sigma_s\(\br,\bo \cdot \bo',E'\rightarrow E\)
\Psi(\br,\bo',E')dE'd\bo' + Q(\br,\bo,E)
\end{equation}
Like we said before, we will replace this equation by the
Boltzmann-Fokker-Planck equation (BFP). The idea is to decompose the higly forward
peaked scattering cross section into a sum of a forward-peaked cross section
and a smooth cross section. The BFP equation is given by (the variables are
omitted for brievety) \cite{morel_96} :
\begin{equation}
\bo \cdot \bn \Psi + \Sigma_t \Psi = \int_{4\pi}\int_0^{\infty} \Sigma_s\(\bo
\cdot \bo',E'\rightarrow E\) \Psi(\bo',E')dE'd\bo' + \frac{\alpha}{2}
\(\frac{\partial}{\partial \mu} \(1-\mu^2\) \frac{\partial \Psi}{\partial \mu}
+ \frac{1}{1-\mu^2} \frac{\partial^2 \Psi}{\partial \phi}\) +
\frac{\partial S\Psi}{\partial E}  + Q
\end{equation}
where the second and the third term of the right-hand-side changes the direction of 
the particles and the fourth term changes the energy of the particles. $\alpha$ is 
the restricted momentum transfert :
\begin{equation}
\alpha(E) = 2 \pi \int_0^E \int_{-1}^1 \Sigma_s(E\rightarrow E',\mu_0) (1-\mu_0) 
d\mu_0 dE
\end{equation}
and $S$ is the restricted stopping power :
\begin{equation}
S(E) = 2\pi \int_0^{E} \int_{-1}^{1} \Sigma_s(E\rightarrow E') \(E-E'\)d\mu_0 dE'
\end{equation}
It is worth to notice that the term
$\frac{\alpha}{2}\(\frac{\partial}{\partial \mu}\(1-\mu^2\)\frac{\partial
\Psi}{\partial \mu}+\frac{1}{1-\mu^2}\frac{\partial^2\Psi}{\partial \phi}\)$ is very 
close to 0 when $\mu \neq 1$. Thus, this term can be replace by
$c\delta(1-\mu) \Psi$. So the equation can be rewritten as : 
\begin{equation}
\bo \cdot \bn \Psi + \Sigma_t \Psi = \int_{4\pi}\int_0^{\infty} \Sigma_s\(\bo
\cdot \bo',E'\rightarrow E\) \Psi(\bo',E')dE'd\bo' + c \delta(1-\mu) \Psi +
\frac{\partial}{\partial E}\beta \Psi + Q
\end{equation}
The 2 extra-terms, in the previous equation, can be either treated explicitly
or they can be treated in the cross sections. That was done in the CEPXS code
\cite{cepxs}. In this work, we will use the cross section produced by CEPXS to
compute the coupled photon-electron transport.\\
\red{Parler de Galerkin mais besoin du papier de Morel.}


% Memory
\section{Memory}
In the previous section, we showed that we must use the Galerkin to solve
(\ref{solved}) correctly. That is a huge problem problem for the memory
consumption. Usually there are much less flux moments than there are angular
fluxes. Thus, the flux is store by storing its moments. However, like we
showed before, the Galerkin quadrature requires the number of moments to be
equal to the number of directions. In \cite{mem}, the authors used a Legendre
expansion of the scattering cross section lower than the one required by the
Galerkin quadrature. They have shown a good agreement between their results and
Monte-Carlo. In this work, we will study more deeply the effects of changing
the order of the Legendre expansion while keeping a larger number of angular
fluxes. To do that, after building the matrices $M$ and $D$ using the Galerkin 
quadrature, we will truncate them to keep only the lower moments. Thus, $M$
and $D$ will be rectangular instead of square. We will compare the results obtained 
with our code with the results given by CEPXS/ONELD \cite{cepxs}. We will modify for 
3 materials (water, Al and Au) the Legendre expansion.


% Results
\section{Results}
In this section, we compare the results that we obtain for the three materials 
with ONELD\cite{cepxs}. ONELD is a code that solves the  
equation (\ref{solved}) in one dimension using the cross sections generated by CEPXS. 
To compare the results produced by the two codes, we use equivalent angular 
discretization, Gauss-Legendre for ONELD, and Gauss-Legendre-Chebyshev for our
code. The Gauss-Legendre-Chebyshev (GLC) quadrature consists of a Gauss-Legendre
quadrature for the polar angle and a Chebyshev quadrature for the azimuthal
angle. The $n$-points Chebyshev quadrature uses $n$ points equally spaced
between 0 and $2\pi$. If the solution is independent of the azimuthal angle,
the GLC quadrature is equivalent to the one dimensional Gauss-Legendre
quadrature. The results of ONELD gives only the average dose on a cell while
the results of our code gives the dose at a given point. To compare the value of 
the dose at a given point, we will interpolate the value given by ONELD.

\subsection{Water}
Here, we use a $S_{12}$ quadrature. The medium is 5 cm thick and there is an 
incoming flux of photons. The direction of the flux is chosen to be the most 
normal direction of the quadrature and the value of the flux on the boundary is 1 
$\frac{photon}{cm^2s}$. The source of photons has an energy of 20 MeV and we use 
a cut-off energy of 0.01 MeV. Every particle that has an energy lower than 
0.01 MeV is assumed to deposit all its energy without moving further.\\
In the next table, we show the dose computed every centimeter using a
different scattering \hbox{order :}
\begin{table}[H]
\begin{center}
\caption{Dose in $\frac{MeV}{g}$ for different scattering orders}
\begin{tabular}{|c|c|c|c|c|c|c|}
\hline
Position & ONELD & order = 13 & order = 11 & order = 9 & order = 7 & order = 5 \\
\hline
0 & 4.92645e-3 &  &  &  & 6.99349e-3 &  \\
1 & 5.68274e-2 &  &  &  & 5.36101e-2 &  \\
2 & 1.05076e-1 &  &  &  & 1.04073e-1 &  \\
3 & 1.47528e-1 &  &  &  & 1.47958e-1 &  \\
4 & 1.82864e-1 &  &  &  & 1.81629e-1 &  \\
5 & 1.85907e-1 &  &  &  & 1.56533e-1 &  \\
\hline
\end{tabular}
\end{center}
\end{table}     

\subsection{Aluminium}
The setup of this problem, is the same as that of the previous one except now 
the medium is composed of aluminium. In the following table, we show the dose 
computed every centimeter using a different scattering order :
\begin{table}[H]
\begin{center}
\caption{Dose in $\frac{MeV}{g}$ for different scattering orders}
\begin{tabular}{|c|c|c|c|c|c|c|}
\hline
Position & ONELD & order = 13 & order = 11 & order = 9 & order = 7 & order = 5 \\
\hline
0 & 0.015003 & 0.0146404 & 0.0146402 & 0.0141156 & 0.0166718 & 0.0087276 \\
1 & 0.169908 & 0.1697907 & 0.1697890 & 0.1698012 & 0.1695422 & 0.1728969 \\
2 & 0.275425 & 0.2752749 & 0.2752729 & 0.2753298 & 0.2742052 & 0.2775113 \\
3 & 0.316307 & 0.3161310 & 0.3161307 & 0.3165279 & 0.3145572 & 0.3199176 \\
4 & 0.312283 & 0.3120572 & 0.3120566 & 0.3125514 & 0.3104412 & 0.3153226 \\
5 & 0.233048 & 0.2087631 & 0.2087634 & 0.2094064 & 0.2061794 & 0.2154810 \\
\hline
\end{tabular}
\end{center}
\end{table}     
The agreement between ONELD and our code using the full order is excellent.
The differences on the border of the domain are larger, than the differences
in the middle of the domain, due to the fact that
the dose varies quickly near the border. Because of this, the interpolation 
used to find the value at a given point by ONELD is not very accurate.\\ 
We see that if we except the results on the border of the domain, the results 
obtained by using a $P_5$ order for the scattering are very close to the ones 
using the
full order. The reason is that the high order moments are very small. Below,
we show several moments for different groups. The abscissa is the distance in
centimeter and the ordinate is the value of the flux in $\frac{particles}{cm^2
s}$.
\begin{figure}[H]
\begin{minipage}[b]{0.45\linewidth}
\centering
\includegraphics[width=\linewidth]{./images/al/group_0_moment_0}
\caption{Scalar flux in the first photons group}
\end{minipage}
\hspace{0.5cm}
\begin{minipage}[b]{0.45\linewidth}
\centering
\includegraphics[width=\linewidth]{./images/al/group_39_moment_0}
\caption{Scalar flux in the last electrons group}
\end{minipage}
\end{figure}

\begin{figure}[H]
\begin{minipage}[b]{0.45\linewidth}
\centering
\includegraphics[width=\linewidth]{./images/al/group_0_moment_30}
\caption{Moment $P_5^0$ of the flux in the first photons group}
\end{minipage}
\hspace{0.5cm}
\begin{minipage}[b]{0.45\linewidth}
\centering
\includegraphics[width=\linewidth]{./images/al/group_39_moment_30}
\caption{Moment $P_5^0$ of the flux in the last electrons group}
\end{minipage}
\end{figure}

\begin{figure}[H]
\begin{minipage}[b]{0.45\linewidth}
\centering
\includegraphics[width=\linewidth]{./images/al/group_0_moment_142}
\caption{Moment $P_{11}^0$ of the flux in the first photons group}
\end{minipage}
\hspace{0.5cm}
\begin{minipage}[b]{0.45\linewidth}
\centering
\includegraphics[width=\linewidth]{./images/al/group_39_moment_142}
\caption{Moment $P_5^0$ of the flux in the last electrons group}
\end{minipage}
\end{figure}

\subsection{Gold}
The setup of this problem is the same as the others but now the medium is 
made of gold. In the next table, we show the dose computed every centimeter 
using a different scattering order :
\begin{table}[H]
\begin{center}
\caption{Dose in $\frac{MeV}{g}$ for different scattering orders}
\begin{tabular}{|c|c|c|c|c|c|c|}
\hline
Position & ONELD & order = 13 & order = 11 & order = 9 & order = 7 & order = 5 \\
\hline
0 & 0.11675 &  &  & 0.097155 & 0.097073 & 0.097326 \\
1 & 0.41799 &  &  & 0.420871 & 0.420988 & 0.421102 \\
2 & 0.16284 &  &  & 0.164946 & 0.165017 & 0.164605 \\
3 & 0.06407 &  &  & 0.065300 & 0.065292 & 0.065214 \\
4 & 0.02566 &  &  & 0.026304 & 0.026295 & 0.026318 \\
5 & 0.00847 &  &  & 0.006083 & 0.006075 & 0.006112 \\
\hline
\end{tabular}
\end{center}
\end{table}                



% Reordering
\section{REORDERING}
The second method to decrease memory required to solve the transport of
photons-electrons involves reordering the energy groups to simplify the
scattering cross-section matrix. When CEPXS generates the cross sections,
it first writes all the cross sections for one particle type, and then, the cross
sections for the other particle type into the scattering matrix. The
energy range is the same for the two particles. The cross section matrix looks like :
\begin{equation}
\Sigma = 
\begin{pmatrix}
PP & EP\\
PE & EE
\end{pmatrix}
\end{equation}
where PP and EE are lower triangular matrices which represent the
scattering for each particle type. The two matrices are lower triangular
because only down scattering is allowed. The cut-off energy used in
radiotherapy approximations forbids the 
thermalization of particles. Matrix EP represents the creation of photons by
electrons through bremsstrahlung production and fluorescence production.
Matrix PE represents the creation of electrons by photons through photoelectric 
effect, Compton scattering, pair electron production, and Auger production 
following photoionization. Now it is important to notice that because 
of energy conservation, a particle can only create a particle which has a 
energy equal or lower than its own. An example of the transfer
of two photon groups and four electron groups can be seen in \hbox{Figure
\ref{joli} :}
\begin{figure}[H]
\begin{center}
\includegraphics[height=7cm]{group.png}
\caption{\bf{Transfer between the different groups}}
\label{joli}
\end{center}
\end{figure}
The pattern of the scattering matrix is as follows :
\begin{equation}
\Sigma =
\begin{pmatrix}
x & 0 & \vline & x & x & 0 & 0\\
x & x & \vline & x & x & x & x\\
\hline     
x & 0 & \vline & x & 0 & 0 & 0\\
x & 0 & \vline & x & x & 0 & 0\\
x & x & \vline & x & x & x & 0\\
x & x & \vline & x & x & x & x\\
\end{pmatrix}
\end{equation}
We can see that there is no upscattering to the first group of photons, the
first and the second group of electrons coming from the second group of
photons, and the third or the fourth group of electrons (see Figure
\ref{joli}). If we reorder the groups using for the first group \hbox{set :} 
photon group 1, electron group 1, electron group 2 and for the second group 
\hbox{set :} photon group 2, electron group 3, electron group 4. The pattern 
of the scattering matrix looks \hbox{like :}
\begin{equation}
\Sigma =
\begin{pmatrix}
x & x & x & \vline & 0 & 0 & 0\\
x & x & 0 & \vline & 0 & 0 & 0\\
x & x & x & \vline & 0 & 0 & 0\\
\hline      
x & x & x & \vline & x & x & x\\
x & x & x & \vline & x & x & 0\\
x & x & x & \vline & x & x & x\\
\end{pmatrix}
\end{equation}
Now, we see that the matrix is block lower triangular and we can solve the
transport problem by solving two problems with only three groups each. 
We can solve the first three groups without solving the last three groups since 
there is no upscattering coming from these. Then, we can solve the last three 
groups, with the first three groups hidden in the source term. 
If there are more than two group sets, the fixed source,
which now contains the scattering source, can be saved on an auxiliary
memory, like for example on the hard disk. Only the source for the groups of the
group set that we are solving are needed in memory.
To know how many groups we need to gather from each particle 
in each group set, we can use a simple rule. First, we define the number of
photon groups as $n_p$, the number of electron groups as $n_e$ and the 
greatest common divisor between these two numbers as $gcd$. The number of
photon groups to put in a group set is $\frac{n_p}{gcd}$ and the number 
of electron groups to put in a group set is $\frac{n_e}{gcd}$. Instead 
of solving one problem of $n$ groups, we can solve $gcd$ problems of 
$\frac{n}{gcd}$ groups. Notice that we assumed that CEPXS generates groups of
equal energy per groups. This is only one of the two ways used by CEPXS to
generate the cross sections. For the other one, the energy per group varies 
logarithmically. In this case, if we assume that $n_e \geq n_p$, we will have 
$n_p$ group sets, $n_p-1$ group sets containing each one electron group and one 
photon group and one group containing the photon group of lowest energy and 
the $n_e-(n_p-1)$ electron groups of lowest energy.\\
Reordering the groups can also decrease the number of source iteration or
GMRES iterations. These two methods are used to solve the linear system 
created by the discretized transport equation of (\ref{solved}).
Below, we compare the number of iterations needed to
solve a problem similar to the one described in the previous section with and
without reordering. The domain is
made of aluminium, we use a $S_{12}$ angular discretization and $P_5$ expansion
for the scattering cross section. We use linear discontinuous finite elements 
for the spatial discretization. The range of energy studied is [0.01MeV,10MeV]. 
There are 180 cells and we have 15 groups of photon and 25 groups of electron.
We compare three methods : the standard multigroup iterative scheme, the
standard multigroup iterative scheme with reordering and the blocked
multigroup iterative scheme. The last scheme solves eight multigroup
transport problems, each of them having only five groups.
We obtain the following result :
\begin{table}[H]
\begin{center}
\caption{\bf{Comparison of the number of iterations with and without group
reordering}}
\begin{tabular}{|c|c|c|c|}
\hline
&\multicolumn{2}{c|}{With reordering} & Without reordering\\
\hline
&Modified multigroup & Standard multigroup & Standard multigroup\\
\hline
Total inner iterations & 4232 & 6956 & 7398\\ 
Total outer iterations & 17 $\approx$ 2 per block & 5 & 5\\
\hline
\end{tabular}
\end{center}
\end{table}
Reordering the groups decreases the number of inner iterations by 6\% while
the blocked scheme reduces the number of inner iterations by 42\% compare to the
standard multigroup scheme.


% Conclusions
\section{CONCLUSIONS}
In this paper, we have shown that it is not necessary to keep the full order of
the scattering cross sections to obtain an accurate result while solving
photon-electron transport. Truncating the Galerkin quadrature or equivalently
truncating the scattering expansion while keeping a large number of directions 
allows us to significantly decrease the memory requirements. We also presented a 
method to advantageously reorder the energy groups in the scattering matrix. 
The groups can be reordered in such a way that a problem with $n_p$ groups of 
photons and $n_e$ groups of electrons can be rewritten as $gcd$ problems of 
$\frac{n_p+n_e}{gcd}$ groups, where $gcd$ is the greatest common divisor 
between $n_p$ and $n_e$. The reordering also decreases the number of iterations 
needed to solve the multigroup transport equation. These two methods can 
significantly decrease the memory needed to solve coupled photon/electron 
transport problems.


\bibliographystyle{unsrt}
\bibliography{biblio}

\end{document}
