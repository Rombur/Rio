\section{Results}
In this section we will show the results that we get for the 3 materials. We
will compare the results we get with our 3D code and CEPXS/ONELD, which is a
1D code. 
\subsection{Water}
\subsection{Aluminium}
For this example, we are using a $S_{10}$ Galerkin Gauss-Legendre-Chebyshev 
quadrature. The medium is 5 cm thick and there is an incoming flux of photons.
The source of photons has an energy of 20 MeV and we use a cut-off energy of
0.01 MeV. Thus, every particles which has an energy lower than 0.01 MeV is
assumed to depose all its energy without moving anymore.
\subsection{Gold}
