\section{Memory issues}
In the previous section, we showed that we must use a Galerkin quadrature to solve
(\ref{solved}) correctly. This may be a significant issue in terms of memory
requirements.  
For neutron transport, the scattering is less anisotropic than for electron
transport, and therefore, fewer flux moments than angular 
fluxes are needed. This is the reason why the information is stored as flux moments. 
However, Galerkin quadratures require the number of moments 
to be equal to the number of directions. This requirement is very restrictive and it 
is important to explore whether it is possible to decrease the number of moments 
while keeping a larger number of directions. In \cite{mem}, the authors used a 
Legendre expansion of the scattering cross section lower than that required by the 
Galerkin quadrature. They have shown a good agreement between their results 
and Monte-Carlo simulations. In the next section, we will study the effects of changing 
the order of the Legendre expansion while keeping a larger number of angular 
fluxes than the order of the scattering expansion. We proceed  by building 
the matrices $M$ and $D$ using the Galerkin quadrature, then truncate them 
to keep only a few moments. Therefore, $M$ and $D$ will be rectangular instead 
of square matrices. We show the results for two materials (Al and Au) 
and for different Legendre expansions.
