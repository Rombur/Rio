\section{Memory}
In the previous section, we showed that we must use the Galerkin to solve
(\ref{solved}) correctly. That is a huge problem problem for the memory
consumption. Usually there are much less flux moments than there are angular
fluxes. Thus, the flux is store by storing its moments. However, like we
showed before, the Galerkin quadrature requires the number of moments to be
equal to the number of directions. In \cite{mem}, the authors used a Legendre
expansion of the scattering cross section lower than the one required by the
Galerkin quadrature. They have shown a good agreement between their results and
Monte-Carlo. In this work, we will study more deeply the effects of changing
the order of the Legendre expansion while keeping a larger number of angular
fluxes. To do that, after building the matrices $M$ and $D$ using the Galerkin 
quadrature, we will truncate them to keep only the lower moments. Thus, $M$
and $D$ will be rectangular instead of square. We will compare the results obtained 
with our code with the results given by CEPXS/ONELD \cite{cepxs}. We will modify for 
3 materials (water, Al and Au) the Legendre expansion.
