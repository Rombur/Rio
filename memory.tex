\section{Memory issue}
In the previous section, we showed that we must use the Galerkin to solve
(\ref{solved}) correctly. That is a huge problem for memory consumption. Usually 
there are much less flux moments than there are angular fluxes. That is the
reason why the flux is stored by storing its moments. However, as we showed 
before, the Galerkin quadrature requires the number of moments to be equal to the 
number of directions. This requirement is very restrictive and it is important to 
see if it is acceptable to decrease the number of moments while keeping a higher 
number of direction. In \cite{mem}, the authors used a Legendre expansion of the 
scattering cross section lower that required by the Galerkin quadrature. They have 
shown a good agreement between their results and Monte-Carlo. In the next section, 
we will study more deeply the effects of changing the order of the Legendre 
expansion while keeping a larger number of angular fluxes. To do that, after 
building the matrices $M$ and $D$ using the Galerkin quadrature, we will 
truncate them to keep only a few moments. Therefore, $M$ and $D$ will be rectangular 
instead of square. We will show the results for three materials (water, Al and Au) 
and different Legendre expansions.
