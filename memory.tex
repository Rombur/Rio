\section{Memory issue}
In the previous section, we showed that we must use the Galerkin to solve
(\ref{solved}) correctly. This is a significant problem for memory consumption.  
For neutron transport, the scattering is less anisotropic than for electron
transport and, therefore, there are much fewer flux moments than there are angular 
fluxes. This is the reason why the flux is stored by storing its moments. 
However, as we showed before, the Galerkin quadrature requires the number of moments 
to be equal to the number of directions. This requirement is very restrictive and it 
is important to explore whether it is possible to decrease the number of moments 
while keeping a larger number of direction. In \cite{mem}, the authors used a 
Legendre expansion of the scattering cross section lower than that required by the 
Galerkin quadrature. They have shown a good agreement between their results 
and Monte-Carlo. In the next section, we will study the effects of changing 
the order of the Legendre expansion while keeping a larger number of angular 
fluxes than the order of the scattering expansion. We proceed  by building 
the matrices $M$ and $D$ using the Galerkin quadrature, then truncating them 
to keep only a few moments. Therefore, $M$ and $D$ will be rectangular instead 
of square. We will show the results for three materials (water, Al and Au) 
and different Legendre expansions.
