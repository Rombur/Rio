\documentclass[11pt,letter,twoside]{mc2011}

\globalabst

\begin{document}

\title{TECHNIQUES TO REDUCE MEMORY REQUIREMENTS FOR COUPLED PHOTON-ELECTRON TRANSPORT.}

\author{
    \textbf{Bruno Turcksin, Jean Ragusa, and Jim Morel}\\
    Texas A\&M University, Department of Nuclear Engineering\\
    College Sation, Texas 77843-3133\\
    turcksin@neo.tamu.edu; ragusa@ne.tamu.edu; morel@ne.tamu.edu
}

\maketitle

\thispagestyle{empty}

\section*{ABSTRACT}

\small
 In this work, we present two methods to decrease memory needs while solving 
the photon-electron transport equation. The coupled transport of electrons and
photons is of importance in radiotherapy because it describes the interactions of
X-rays with matter. One of the issues of discretized electron transport is that the
electron scattering is highly forward peaked. A common approximation is to
represent the peak in the scattering cross section by a Dirac distribution.
This is convenient, but the integration over all angles of this distribution requires the
use of Galerkin quadratures. By construction these quadratures impose that
the number of flux moments be equal to the number of directions 
(number of angular fluxes), which is very demanding in terms of memory. In this study, 
we show that even if the number of moments is not as large as the number 
of directions, an accurate solution can be obtained when using Galerkin
quadratures. Another method to decrease the 
memory needs involves choosing an appropriate reordering of the energy
groups. 
We show in this paper that an appropriate alternation of photons/electrons groups 
allows us to rewrite one transport problem of $n$ groups as $gcd$ successive 
transport problems of $\frac{n}{gcd}$ groups where $gcd$ is the greatest common 
divisor between the number of photon groups and the number of electron groups.
\keywords{Galerkin quadrature, photon-electron transport, radiotherapy}


\end{document}
